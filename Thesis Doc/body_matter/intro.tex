\chapter{Εισαγωγή}
\InitialCharacter{Η} ιατρική απεικόνιση αποτελεί έναν από τους πιο κρίσιμους πυλώνες της σύ\-γχρονης κλι\-νικής πρακτικής,
καθώς παρέχει μη επεμβατική πρόσβαση σε πληροφορίες για την ανατομία και τη λειτουργία βιολογικών δομών. Στον χώρο της μαιευτικής,
η αξιόπιστη εκτίμηση της κύησης επηρεάζει άμεσα τη λήψη κλινικών αποφάσεων και την ασφάλεια μητέρας και εμβρύου.
Σε αυτό το πλαίσιο, ο πλακούντας αναδεικνύεται σε όργανο-«κλειδί», καθώς μεσολαβεί στην ανταλλαγή οξυγόνου και θρεπτικών συστατικών,
στη ρύθμιση ορμονικών και ανοσολογικών μηχανισμών, και συνολικά στην ομαλή εξέλιξη της κύησης \cite{torrentsbarrena2019fetalreview}.

Παρότι ο υπέρηχος αποτελεί την κύρια μέθοδο προγεννητικού ελέγχου, υπάρχουν περιπτώσεις όπου η απεικόνιση είναι δύσκολη ή ανεπαρκής
(π.χ. λόγω θέσης πλακούντα, σωματοδομής, περιορισμών οπτικού πεδίου ή ανάγκης λεπτο\-μερέστερης απεικόνισης). Η Μαγνητική Τομογραφία
\en{(Magnetic Resonance Imaging, MRI)} λειτουργεί συμπληρωματικά,
προσφέροντας αυξημένη αντίθεση μαλακών ιστών και δυνατότητα τρισδιάστατης απεικό\-νισης.
Επιπλέον η \en{fetal MRI} μπορεί να αναδείξει καταστάσεις όπως εγκεφαλικές ανωμαλίες, παθολογίες των πνευμόνων και μειωμένη ποσότητα αμνιακού υγρού,
οι οποίες συχνά δεν ανιχνεύονται με τον υπέρηχο \cite{DBLP:journals/vciba/JittouFR25}.

Η τμηματοποίηση, δηλαδή ο ακριβής διαχωρισμός του πλακούντα από το υπόλοιπο απεικονιστικό υπόβαθρο,
αποτελεί θεμελιώδες βήμα για υπολογισμούς όπως:
\begin{itemize}
  \item εκτίμηση όγκου και σχήματος,
  \item εξαγωγή μορφολογικών/υφικών χαρακτηριστικών,
  \item ανάλυση χωρικής ετερογένειας και
  \item τυποποιημένη σύγκριση μεταξύ εξετάσεων, ασθενών ή πρωτοκόλλων.
\end{itemize}

Στην πράξη, η χειροκίνητη τμηματοποίηση είναι χρονοβόρα,
η ποιότητα της εξαρτάται \-απο την εμπειρία του παρατηρητή και εμφανίζει σημαντικές παραλλαγές με βάση τον παρατηρητή \cite{DBLP:journals/vciba/JittouFR25},
ιδίως όταν τα όρια του πλακούντα είναι ασαφή λόγω θορύβου,
κίνησης ή περιορισμένης αντίθεσης.

Τα τελευταία χρόνια, οι μέθοδοι Βαθιάς Μάθησης \en{(Deep Learning)} έχουν επιφέρει ση\-μαντική πρόοδο σε προβλήματα τμηματοποίησης ιατρικών εικόνων,
επιτρέποντας αυτόματη εξαγωγή ισχυρών αναπαραστάσεων και εκμάθηση πολύπλοκων χωρικών μοτίβων \cite{Litjens_2017}. Αρχι\-τεκτονικές τύπου
\en{U-Net} και οι νεότερες παραλλαγές τους αποτελούν σημείο αναφοράς στον χώρο \cite{ronneberger2015unetconvolutionalnetworksbiomedical}.
Παράλληλα, υβριδικά μοντέλα που ενσωματώνουν μηχανισμούς προσοχής και~\en{Transformer-based blocks} έχουν δειξει υψηλές επιδόσεις,
καθώς μπορούν να μοντελοποιούν αποτελεσματικότερα μακρινές χωρικές εξαρτήσεις και συσχετίσεις μεταξύ απομακρυσμένων περιοχών της εικόνας/όγκου.
έχουν δώσει ιδιαίτερα ισχυρά αποτελέσματα σε πλήθος απεικονιστικών εφαρμογών
\cite{Litjens_2017},\cite{ronneberger2015unetconvolutionalnetworksbiomedical}.

Πιο πρόσφατα, εναλλακτικά υποδείγματα ακολουθιακής μοντελοποίησης όπως τα \en{Structured State Space Models (SSMs)} και το
\en{Mamba} προτείνονται ως αποδοτικότερες επιλογές για την αποτύπωση μακρινών εξαρτήσεων με γραμμική πολυπλοκότητα  ως προς το μήκος της ακολουθίας \cite{gu2023mamba}.
Στο πλαίσιο της τρισδιάστατης ιατρικής τμηματοποίησης, μοντέλα όπως το \en{SegMamba} επιχειρούν να αξιοποιήσουν αυτή τη φιλοσοφία για αποδοτικότερη επεξεργασία \cite{xing2024segmambalongrangesequentialmodeling}.

Παρόλα αυτά, η τμηματοποί\-ηση πλακούντα σε \en{MRI} παραμένει απαιτητικό πρόβλημα:
\-ο πλακούντας εμφανίζει μεγάλη διακύμανση ως προς μέγεθος, σχήμα και θέση καθ'όλη τη διάρκεια της κύησης,
ενώ παράγοντες που σχετίζονται με την υγεία της μητέρας μπορούν να επιτείνουν τη μεταβλητότητα και να δυσχεραίνουν την εκπαίδευση, μειώνοντας την ακρίβεια, ιδίως στον υπέρηχο.
Παθολογίες της μητέρας όπως ο σακχαρώδης διαβήτης ή λοιμώξεις μπορεί να οδηγήσουν σε μη ομαλή εξόγκωση ή σμίκρυνση του πλακούντα \cite{Oppenheimer2018MagneticRI}.
Τέλος, η εντόπιση του (πρόσθια, οπίσθια, πλάγια ή θόλος της μήτρας) επηρεάζει την απεικονιστική του εμφάνιση, για παράδειγμα, οι πρόσθιοι πλακούντες συχνά εμφανίζουν καλύτερη αντίθεση στον υπέρηχο, ενώ οι οπίσθιοι μπορεί να αποκρύπονται μερικώς απο το έμβρυο, προκαλώντας σκιαστικά φαινόμενα. \cite{ZIMMER2023102639},\cite{fetalMRIwhatsnew2023}.
