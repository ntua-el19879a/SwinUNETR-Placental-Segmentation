\chapter{Εισαγωγή}
\InitialCharacter{Η} ιατρική απεικόνιση αποτελεί έναν απο τους πιο κρίσιμους πυλώνες της σύ\-γχρονης κλι\-νικής πρακτικής,
καθώς παρέχει μη επεμβατική πρόσβαση σε πληροφορίες για την ανατομία και τη λειτουργία βιολογικών δομών.\  Στον χώρο της μαιευτικής,
η αξιόπιστη εκτίμηση της κύησης επηρεάζει άμεσα τη λήψη κλινικών αποφάσεων και την ασφάλεια μητέρας και εμβρύου. 
Σε αυτό το πλαίσιο, ο πλακούντας αναδεικνύεται σε όργανο-«κλειδί», καθώς μεσολαβεί στην ανταλλαγή οξυγόνου και θρεπτικών συστατικών,
στη ρύθμιση ορμονικών και ανοσολογικών μηχανισμών, και συνολικά στην ομαλή εξέλιξη της κύησης \cite{torrentsbarrena2019fetalreview}.

Παρότι ο υπέρηχος αποτελεί την κύρια μέθοδο προγεννητικού ελέγχου,  υπάρχουν περιπτώσεις όπου η απεικόνιση είναι δύσκολη ή ανεπαρκής 
(π.χ.\ λόγω θέσης πλακούντα, σωματοδομής, περιορισμών οπτικού πεδίου ή ανάγκης λεπτο\-μερέστερης απεικόνισης).\  Η Μαγνητική Τομογραφία
\en{(Magnetic Resonance Imaging, MRI)} λειτουργεί συμπληρωματικά, προσφέροντας αυξημένη αντίθεση μαλακών ιστών και δυνατότητα τρισδιάστατης απεικόνισης.
Παράλληλα η \en{fetal MRI} με ιδιαίτερη αξία σε καταστάσεις όπου απαιτείται λεπτομερής αξιολόγηση της μορφολογίας και της χωρικής σχέσης του πλακούντα με παρακείμενες δομές.

Ωστόσο, η αξιοποίηση της \en{MRI} σε κλινικό ή ερευνητικό περιβάλλον προϋποθέτει συχνά την ποσοτικοποίηση και την αντικε
ιμενική περιγραφή ευρημάτων. Η τμηματοποίηση, δη\-λαδή ο ακριβής διαχωρισμός του πλακούντα απο το υπόλοιπο απεικονιστικό υπόβαθρο, αποτελεί θεμελιώδες βήμα για υπολογισμούς όπως:
\begin{itemize}
  \item εκτίμηση όγκου και σχήματος,
  \item εξαγωγή μορφολογικών/υφικών χαρακτηριστικών,
  \item ανάλυση χωρικής ετερογένειας και
  \item τυποποιημένη σύγκριση μεταξύ εξετάσεων, ασθενών ή πρωτοκόλλων.
\end{itemize}

Στην πράξη,\  η χειροκίνητη τμηματοποίηση είναι χρονοβόρα, η ποιότητα της εξαρτάται απο την εμπειρία του παρατηρητή και\ εμφανίζει σημαντικές παραλλαγές με βάση τον παρατηρητή, ιδίως όταν τα όρια του πλακούντα είναι ασαφή λόγω θορύβου, κίνησης ή περιορισμένης αντίθεσης.

Τα τελευταία χρόνια, οι μέθοδοι Βαθιάς Μάθησης \en{(Deep Learning)} έχουν επιφέρει σημαντική πρόοδο σε προβλήματα τμηματοποίησης ιατρικών εικόνων, επιτρέποντας αυτόματη εξαγωγή πλουσίων αναπαραστάσεων και εκμάθηση πολύπλοκων χωρικών μοτίβων. Αρχιτεκτονικές τύπου
\en{U-Net} και οι νεότερες παραλλαγές τους, καθώς και μοντέλα που ενσωματώνουν μηχανισμούς προσοχής ή \en{Transformer-based blocks}, έχουν δώσει ιδιαίτερα ισχυρά αποτελέσματα σε πλήθος απεικονιστικών εφαρμογών
\cite{ronneberger2015unetconvolutionalnetworksbiomedical},
\cite{Litjens_2017}. Παρόλα αυτά, η τμηματοποίηση πλακούντα σε \en{MRI} παραμένει απαιτητικό πρόβλημα:\  ο πλακούντας εμφανίζει μεγάλη διακύμανση ως προς μέγεθος, σχήμα και θέση, ενώ οι εξετάσεις επηρεάζονται απο κίνηση μητέρας/εμβρύου, ανομοιόμορφη κατανομή της έντασης στην εικόνα και διαφορές στα πρωτόκολλα λήψης.
\cite{Alansary2016PlacentaMotionMRI}, \cite{Wang2016SlicSeg}, \cite{Tustison2010N4ITK}.

Η παρούσα διπλωματική εστιάζει στη μελέτη και υλοποίηση ενός ολοκληρωμένου πλαισίου αυτόματης τμηματοποίησης πλακόυντα σε \en{MRI}, αξιοποιώντας σύγχρονες αρχιτεκτονικές Βαθιάς Μάθησης και πρότυπες φάσεις προεπεξεργασίας, εκπαίδευσης και αξιολόγησης. Στόχος είναι η τεκμηρίωση της υπεροχής συγκεκριμένων αρχιτεκτονικών μέσω πειραμάτων, με γνώμονα την εξαγωγή αξιόπιστων αποτελεσμάτων που, μακροπρόθεσμα, να μπορούν να αξιοποιηθούν κλινικά.
\section{Αντικείμενο της διπλωματικής}
\section{Οργάνωση του τόμου}
