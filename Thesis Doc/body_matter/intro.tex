\chapter{Εισαγωγή}
\label{ch:intro}

\InitialCharacter{Η} ιατρική απεικόνιση αποτελεί αναπόσπαστο μέρος της σύγχρονης κλινικής πρακτικής και της βιοϊατρικής
έρευνας, καθώς επιτρέπει τη μη επεμβατική παρατήρηση ανατομικών δομών και φυσιολογικών διεργασιών. Πέρα όμως από την
οπτική επιβεβαίωση ή περιγραφική αξιολόγηση, σε πολλές εφαρμογές είναι απαραίτητη η \textit{ποσοτικοποίηση}: η εξαγωγή
μετρήσιμων μεγεθών (π.χ. όγκοι, σχήματα, σχέσεις γειτονικών ιστών), που στη συνέχεια μπορούν να συσχετισθούν με κλινικά
ευρήματα ή να αξιοποιηθούν σε συστήματα υποβοήθησης διάγνωσης.

\section{Συνοπτική Εισαγωγή}
Κεντρικό βήμα για την ποσοτικοποίηση αποτελεί η \textbf{τμηματοποίηση} (\en{segmentation}) μιας ιατρικής εικόνας ή ενός
ογκομετρικού δεδομένου. Με τον όρο τμηματοποίηση εννοούμε την ανάθεση ετικέτας σε κάθε (ογκομετρικό) εικονοστοιχείο
(\en{pixel/voxel}), ώστε να διαχωριστεί η \textit{περιοχή ενδιαφέροντος} (\en{Region of Interest, ROI}) από το υπόλοιπο
υπόβαθρο. Στην πράξη, η τμηματοποίηση επιστρέφει μία \textit{μάσκα} (\en{mask}) με την ίδια διάσταση όπως η είσοδος, η
οποία κωδικοποιεί το ``ανήκει/δεν ανήκει'' (δυαδική τμηματοποίηση) ή/και πολλαπλές κλάσεις (πολυκλασική τμηματοποίηση).
Η τμηματοποίηση είναι η βάση για τον υπολογισμό όγκου, την εκτίμηση μορφολογικών χαρακτηριστικών, την ανάλυση χωρικής
ετερογένειας, αλλά και για πιο σύνθετες ροές εργασίας, όπως εξαγωγή χαρακτηριστικών (\en{radiomics}) ή καθοδήγηση
επεμβάσεων.

Στο πλαίσιο της μαιευτικής απεικόνισης, ιδιαίτερο ενδιαφέρον παρουσιάζει ο \textbf{πλακούντας}, ένα λειτουργικά κρίσιμο
όργανο που συνδέει μητέρα και έμβρυο. Η \textbf{αυτόματη τμηματοποίηση πλακούντα} σε δεδομένα \en{MRI} επιτρέπει
επαναλήψιμες και αντικειμενικές μετρήσεις (π.χ. όγκος/σχήμα), καθώς και τη συστηματική σύγκριση μεταξύ εξετάσεων ή
πρωτοκόλλων. Ωστόσο, η τμηματοποίηση πλακούντα σε \en{MRI} είναι απαιτητικό πρόβλημα: τα δεδομένα συχνά επηρεάζονται από
κίνηση μητέρας/εμβρύου, ανομοιογένειες έντασης, περιορισμένη αντίθεση στα όρια οργάνου-υποβάθρου και μεγάλη διακύμανση
σε μέγεθος, σχήμα και θέση. Προγενέστερες προσεγγίσεις περιλάμβαναν κλασικούς ταξινομητές ή ελάχιστα/μερικώς διαδραστικές
μεθόδους, ενώ έχουν παρουσιαστεί και τρόποι για αντιμετώπιση σε περίπτωση που αλλοιωθεί η εικόνα λόγω κίνησης.
\cite{alan._2016}\cite{Wang2016SlicSeg}\cite{shahedi2021placenta}.

Η χειροκίνητη τμηματοποίηση, αν και θεωρείται σημείο αναφοράς, είναι χρονοβόρα και εμφανίζει
διακύμανση μεταξύ παρατηρητών, κάτι που επηρεάζει τόσο την αξιολόγηση όσο και την κλινική αξιοποίηση. Για τον λόγο αυτό,
τα τελευταία χρόνια κυριαρχούν μέθοδοι \textbf{Βαθιάς Μάθησης} (\en{Deep Learning}) για τμηματοποίηση ιατρικών εικόνων,
με αρχιτεκτονικές τύπου \en{U\!-\!Net} να αποτελούν θεμελιώδη επιλογή \cite{ronneberger2015unet}. Παράλληλα, μοντέλα με
μηχανισμούς \en{attention} και \en{Transformers} έχουν δείξει ιδιαίτερα ισχυρή ικανότητα μοντελοποίησης διευκολύνοντας την συσχέτιση μακρινών εικονοστοιχείων (\en{global context}) \cite{vaswani2017attention,hatamizadeh2022unetr,hatamizadeh2022swinunetr}.
Πρόσφατα, τα \textbf{Μοντέλα Χώρου και Κατάστασης} (\en{State Space Models, SSMs}) και ειδικότερα το \en{Mamba} εισάγουν μία
εναλλακτική προσέγγιση μακράς εμβέλειας, με στόχο υψηλή αποδοτικότητα σε μνήμη και χρόνο \cite{gu2023mamba}, και έχουν
προταθεί υβριδικά μοντέλα τμηματοποίησης όπως το \en{SegMamba} \cite{xing2024segmamba}. 
\textit{(Λεπτομέρειες στα κεφάλαια 2,3,4)}

\section{Αντικείμενο της διπλωματικής}
\label{sec:intro:scope}

Αντικείμενο της παρούσας διπλωματικής εργασίας είναι η \textbf{συστηματική μελέτη, υλοποίηση και συγκριτική αξιολόγηση}
μοντέλων τμηματοποίησης πλακούντα από ογκομετρικά δεδομένα \en{MRI}, με χρήση της βιβλιοθήκης \en{MONAI}
\cite{cardoso2022monai}. Η εργασία επικεντρώνεται σε ένα δίκαιο πειραματικό πρωτόκολλο, όπου τα μοντέλα εκπαιδεύονται και
αξιολογούνται υπό \textit{ίδιες} συνθήκες (ίδιο \en{seed}, ίδια προεπεξεργασία, ίδιοι βελτιστοποιητές, κριτήρια, σχήματα
\en{learning rate}, αριθμός εποχών και συνολικό πείραμα), ώστε οι διαφορές στην επίδοση να αποδίδονται σε μεγαλύτερο βαθμό στην αρχιτεκτονική.

Συγκεκριμένα, εξετάζονται οι παρακάτω αρχιτεκτονικές:
\begin{itemize}
  \item \textbf{Συνελικτικά μοντέλα τύπου \en{U\!-\!Net}}: \en{UNet} \cite{ronneberger2015unet}, \en{DynUNet}
        (δυναμική παραλλαγή σχεδιασμένη με αρχές αυτο-διαμόρφωσης), \en{SegResNet} \cite{myronenko2018segresnet},
        καθώς και \en{AttentionUnet} \cite{oktay2018attentionunet}.
  \item \textbf{Υβριδικά μοντέλα με \en{Transformers}}: \en{UNETR} \cite{hatamizadeh2022unetr} και \en{SwinUNETR}
        \cite{hatamizadeh2022swinunetr}.
  \item \textbf{Μοντέλο βασισμένο σε \en{SSM/Mamba}}: \en{SegMamba} \cite{xing2024segmamba}.
\end{itemize}

Για την αξιολόγηση χρησιμοποιείται κυρίως η \textbf{μετρική \en{Dice}} (γνωστή και ως \en{DSC})
και συμπληρωματικά ο \textbf{\en{Intersection over Union}} (\en{IoU}/\en{Jaccard}). Στόχος δεν είναι απλώς η καταγραφή μιας μέγιστης επίδοσης, αλλά και η \textbf{εξαγωγή συμπερασμάτων}
για τα πλεονεκτήματα/μειονεκτήματα κάθε οικογένειας αρχιτεκτονικών (καθαρά συνελικτικές, \en{Transformer}-ενισχυμένες,
\en{SSM}-βασισμένες) στο συγκεκριμένο πρόβλημα.

\section{Οργάνωση του τόμου}
\label{sec:intro:structure}

Η διάρθρωση της διπλωματικής έχει ως εξής:
\begin{itemize}
  \item Στο \textbf{Κεφάλαιο 2} παρουσιάζονται βασικές έννοιες τμηματοποίησης ιατρικών εικόνων και συνοπτική εισαγωγή στη
        \en{MRI}, με έμφαση στις ιδιαιτερότητες των ογκομετρικών δεδομένων και στα χαρακτηριστικά/τεχνουργήματα που
        επηρεάζουν την τμηματοποίηση πλακούντα.
  \item Στο \textbf{Κεφάλαιο 3} παρουσιάζεται το θεωρητικό υπόβαθρο της Μηχανικής Μάθησης και της Βαθιάς Μάθησης για
        τμηματοποίηση: \en{CNNs}, \en{Transformers} και \en{SSM/Mamba}, καθώς και βασικές επιλογές εκπαίδευσης
        (συναρτήσεις κόστους, βελτιστοποίηση, γενίκευση).
  \item Στο \textbf{Κεφάλαιο 4} αναλύονται οι αρχιτεκτονικές των 7 μοντέλων που χρησιμοποιήθηκαν (όπως περιγράφονται στις
        αντίστοιχες δημοσιεύσεις), καθώς και οι κύριες σχεδιαστικές τους επιλογές σε σχέση με την τμηματοποίηση 3D
        ιατρικών δεδομένων.
  \item Στα επόμενα κεφάλαια (πειραματικό μέρος) παρουσιάζονται το σύνολο δεδομένων, η μεθοδολογία υλοποίησης και
        προεπεξεργασίας, η διαδικασία εκπαίδευσης/αξιολόγησης, τα αποτελέσματα και η συζήτησή τους, και τέλος τα
        συμπεράσματα και οι μελλοντικές επεκτάσεις.
\end{itemize}
