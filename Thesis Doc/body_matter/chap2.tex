\chapter{Θεωρητικό υπόβαθρο}
\InitialCharacter{Σ}το κεφάλαιο αυτό παρουσιάζονται οι βασικές έννοιες που απαιτούνται για την κατανόηση της παρούσας
διπλωματικής,\ σχετικά με την αυτόματη τμηματοποίηση του πλακούντα (\en{Placental segmentation}) σε ογκομετρικά δεδομένα \en{MRI}. Αρχικά ορίζεται το πρόβλημα τμηματοποίησης ιατρικών εικόνων και οι ιδιαιτερότητες του.\ Έπειτα παρουσιάζονται τα συνελικτικά νευρωνικά δίκτυα
(\en{CNNs}) και η αρχιτεκτονική \en{U-Net}, η οποία αποτελεί σημείο αναφοράς για τμηματοποίηση
με βιοϊατρικές εφαρμογές
\cite{ronneberger2015unetconvolutionalnetworksbiomedical}. Τέλος, εισάγεται η αρχιτεκτονικη των \en{Transformers} και ο ρόλος
της στην τμηματοποίηση, με έμφασησε υβριδικά μοντέλα τύπου \en{U-Net} με \en{Transformer} encoder (π.χ. \en{UNETR},
\en{SwinUNETR}, \en{TransUNet})
\cite{vaswani2017attention,hatamizadeh2022unetr,hatamizadeh2022swinunetr, chen2021transunet}.
\section{Αυτόματη τμηματοποίηση εικόνων}

\section{Συνελικτικά Νευρωνικά Δίκτυα}
\subsection{Αρχιτεκτονική \en{UNet}}
\subsection{Περιορισμοί}
\section{\en{Transformers}}
Η αρχιτεκτονική των \en{Transformers} \cite{vaswani2023attentionneed} έφερε συντριπτική βελτίωση σε σχέση με τον
\en{Recurrent neural networks} σχετικά με \en{language modeling} και παρόμοια προβλήματα. Εκτός απο τους περιορισμούς όσ
ο αναφορά την έλλειψη παραλληλισμού στον υπολογισμό, χρειάζοντας έτσι η κατάσταση $\en{h_{t}}$ να αναμένει την ολοκλήρωσ
η του υπολογισμού της κατάστασης $\en{h_{t-1}}$, η προσοχή επιτρέπει σε μοντέλα να κατέχουν και να συσχετίζουν δύο στοιχεία δίχως να επιφέρει υπολογιστικό κόστος η αυξανόμενη μεταξύ τους απόσταση.
\subsection{Χρησιμότητα στην διαδικασία της τμηματοποίησης}

Οι αρχές που διέπουν τα συστήματα ομότιμων κόμβων είναι οι εξής:
\begin{itemize}
\item Η αρχή του μοιράσματος των πόρων.
\item Η αρχή της αυτοοργάνωσης.
\end{itemize}
