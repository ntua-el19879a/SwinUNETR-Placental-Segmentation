% Chapter 2
\chapter{Τμηματοποίηση Ιατρικών Εικόνων και Μαγνητική Τομογραφία}
\label{ch:seg_mri}
\InitialCharacter{Ο} πλακούντας αποτελεί ένα \textit{προσωρινό αλλά κρίσιμο} όργανο της κύησης, το οποίο λειτουργεί ως
επιφάνεια διεπαφής μεταξύ μητέρας και εμβρύου, επιτρέποντας την ανταλλαγή αερίων, θρεπτικών συστατικών και μεταβολισμού ουσιών, ενώ
παράλληλα επιτελεί σημαντικούς ενδοκρινικούς και ανοσολογικούς ρόλους 
\cite{burton2015placenta,maltepe2015forgotten,gude2004growth}. Ο πλακούντας, επίσης, κρατά προσκολλημένο το έμβρυο με το τοίχωμα της μήτρας όπου επιτελεί και αναπνευστική επιφάνεια για το έμβρυο.
Η λειτουργία του πλακούντα σχετίζεται άμεσα με την ομαλή έκβαση της κύησης, και η δυσλειτουργία του έχει
συσχετιστεί με σοβαρές μαιευτικές επιπλοκές, όπως προεκλαμψία, υπολειπόμενη ανάπτυξη εμβρύου (\en{FGR}), αποβολές και ενδομήτριο θάνατο του εμβρύου.

Σε αυτό το πλαίσιο, η ιατρική απεικόνιση -- και ειδικότερα η Μαγνητική Τομογραφία (\en{MRI}) -- μπορεί να υποστηρίξει
την \textit{ποσοτική} αξιολόγηση του πλακούντα. Ωστόσο, για να εξαχθούν αξιόπιστες μετρήσεις (π.\,χ. όγκος, σχήμα,
μορφολογικοί δείκτες ή/και χαρακτηριστι\-κά υφής), απαιτείται πρώτα ο ακριβής διαχωρισμός του πλακούντα από το υπόβα\-θρο:
η \textbf{τμη\-ματοποίηση} (\en{segmentation}). Στην παρούσα εργασία, στόχος είναι η τμηματοποίηση του πλακούντα σε
ογκομετρικά δεδομένα \en{MRI}, δηλαδή η παραγωγή μιας δυαδικής μάσκας (πλακούντας έναντι υποβάθρου) για κάθε εικόνα.

\section{Ο πλακούντας: ανατομία, ανάπτυξη και λειτουργίες}
\label{sec:placenta_phys}

Ο πλακούντας αναπτύσσεται απο την τροφόβλαστη της περιφέρειας της βλαστοκύστης, όπου συμβαίνει η εμφύτευση.
Η εμφύτευση λαμβάνει χώρα επτά με δώδεκα ημέρες μετά τη γονιμοποίηση \cite{https://doi.org/10.1111/aogs.13834}.
Από ανατομικής άποψης, στον πλακούντα διακρίνεται:
\begin{itemize}
  \item το \textbf{χοριακό πέταλο }(\en{chorionic plate}) (εμβρυική πλευρά), από όπου εκφύονται οι χοριακές λάχνες και το δίκτυο αγγείων του εμβρύου,
  \item το \textbf{φθαρτικό πέταλο }(\en{basal plate/decidua}) (μητρική πλευρά), το οποίο συνδέεται με το τοίχωμα της μήτρας. Γίνεται εμφανές αφού κατα τον τοκετό πραγμα\-τοποιείται διαχωρισμός του πλακούντα απο το σώμα της μήτρας μέσω αυτού,
  \item ο \textbf{μεσολάχνιος χώρος }(\en{intervillous space}), όπου κυκλοφορεί μητρικό αίμα, διαχωρίζεται η εμβρυική κυκλοφορία απο τις δεξαμενές του μητρικού αίματος, και πραγμα\-τοποιείται η κύρια μεταφορά οξυγόνου, θρεπτικών συστατικών και μεταβολιτών μεταξύ μητέρας και εμβρύου.
\end{itemize}
Οι χοριακές λάχνες, με υψηλή πυκνότητα σε τροφοβλαστικούς πληθυσμούς (με κυρίαρχο το \en{syncytiotrophoblast}), αποτελούν τον κύριο συντελεστή στην ανταλλαγή συστατικών μεταξύ μητέρας και εμβρύου \cite{gude2004growth,maltepe2015forgotten}.


Καθοριστικό βήμα στην ορθή πλακουντοποίηση είναι η \textbf{αναδιαμόρφωση των μητρι\-κών σπειροειδών αρτηριών}, μέσω
εισβολής εξωλάχνιων τροφοβλάστων (\en{extravillous \-trophoblast}) στον φθαρτό της μήτρας , ώστε να εξασφαλιστεί σταθερή αιμάτωση προς τον
πλακούντα. Η ελατ\-τωματική πλακουντοποίηση έχει προταθεί ως κοινός παθοφυσιολογικός μηχανισμός πίσω από τις διάφορες επιπλοκές της κύησης (π.\,χ. προεκλαμψία, αποκόλληση πλακούντα, \en{fetal growth restriction})
\cite{brosens2011deepplacentation,maltepe2010placenta}.

Λειτουργικά, ο πλακούντας:
\begin{itemize}
  \item εξασφαλίζει \textbf{μεταφορά} οξυγόνου, ηλεκτρολυτών και θρεπτικών ουσιών (νερό, γλυκόζη, αμινοξέα, λιπίδια, βιταμίνες) και συμβάλλει στην \textbf{απομάκρυνση} μεταβολικών προϊ\-όντων όπως ουρεία, ουρικό οξύ και άλλων αποβλήτων \cite{embrplace},
  \item δρα ως \textbf{ενδοκρινικό όργανο} (παραγωγή ορμονών που ρυθμίζουν την κύηση και τον μεταβολισμό της μητέρας),
  \item συμμετέχει στην \textbf{ανοσολογική ανοχή} και στη ρύθμιση της διεπαφής μεταξύ μητέρας και εμβρύου.
\end{itemize}
Συνεπώς, ο πλακούντας δεν είναι απλώς «φίλτρο», αλλά ένα πολυλειτουργικό όργανο με δυναμική
εξέλιξη και προσαρμογή κατά την κύηση \cite{burton2015placenta,maltepe2015forgotten}. 
Η μεταφορά, ωστόσο, δεν αφορά μονο χρήσιμες και απαραίτητες ουσίες. Επιβλαβείς παράγοντες μπορούν 
να περάσουν, όπως ψυχοδραστικές ουσίες, μονοξείδιο του άνθρακα, διάφοροι λοιμώδεις παράγοντες κ.α. \cite{embrplace}.

Η παραπάνω φυσιολογική πολυπλοκότητα εξηγεί γιατί η \textbf{ποσοτική} αξιολόγηση του πλακούντα (δομή/μορφολογία, χωρική
κατανομή, όγκος) αποτελεί ενεργό ερευνητικό πεδίο. Η Μαγνητική (\en{MRI}) προσφέρει υψηλή αντίθεση μαλακών ιστών και τρισδιάστατη
πληροφορία, όμως η ποσοτικοποίηση προϋποθέτει αξιόπιστο \textit{ορισμό της περιοχής του πλακούντα} μέσα στον όγκο, δηλαδή
ακριβής τμηματοποίηση. Η ανάγκη αυτή αποτυπώνεται και σε μελέτες που αναπτύσσουν αυτοματοποιημένους αλγορίθμους για
τμηματοποίηση πλακούντα/μήτρας σε \en{MRI} όπου αναφέρεται άμεσα η σχέση της τμηματοποίησης με ορθή και συνεπή υπολογισμό του όγκου \cite{shahedi2021placenta}.

\section{Η τμηματοποίηση ως υπολογιστικό πρόβλημα}
\label{sec:seg_problem}

Σε ένα τυπικό σενάριο \textbf{δυαδικής τμηματοποίησης}, η είσοδος είναι ένας όγκος
$X \in \mathbb{R}^{H \times W \times D}$ και η έξοδος είναι μια μάσκα
$Y \in \{0,1\}^{H \times W \times D}$ που υποδηλώνει για κάθε \en{voxel} αν ανήκει στην κλάση ενδιαφέροντος.
Η μάθηση διατυπώνεται συνήθως ως \textbf{επιβλεπόμενη} μάθηση: διαθέτουμε σύνολο ζευγών $(X_i, Y_i)$ και επιδιώκουμε
μοντέλο $f_\theta$ ώστε $f_\theta(X_i)\approx Y_i$.

Η τμηματοποίηση διαφέρει από την ταξινόμηση εικόνας \en{(Image Classification)} στο γεγονός ότι η έξοδος είναι \textbf{πυκνή} πρόβλεψη (ανά \en{pixel/voxel}) και όχι μία ετικέτα ανά εικόνα.

Στην \en{3D} τμηματοποίηση, οι υπολογιστικές απαιτήσεις αυξάνονται σημαντικά σε σχέση με το πλήθος των \en{voxels}. Έτσι, συχνά
χρησιμοποιούνται στρατηγικές \textbf{\en{patch-based}} εκπαίδευσης και \textbf{\en{sliding-window inference}}
(\textit{στο Κεφάλαιο \ref{ch:ml_dl_models} αναλύονται οι αντίστοιχες πρακτικές}), έτσι ώστε να είναι
εφικτή η εκ\-παίδευση/πρόβλεψη σε μνήμη \en{GPU}.


\section{Ιδιαιτερότητες τμηματοποίησης πλακούντα σε \en{MRI}}
\label{sec:placenta_specifics}

Η τμηματοποίηση πλακούντα σε \en{MRI} είναι απαιτητική για λόγους που σχετίζονται τόσο με τη φυσιολογία όσο και με τη
διαδικασία λήψης:
\begin{itemize}
  \item \textbf{Μεγάλη μορφολογική μεταβλητότητα}: μέγεθος, σχήμα και θέση του πλακούντα μεταβάλλονται σημαντικά.
  \item \textbf{Περιορισμένη αντίθεση} σε ορισμένες ακολουθίες, ιδιαίτερα στα όρια πλακούντα και παρακείμενων ιστών.
  \item \textbf{Ανισορροπία κλάσεων}: στα περισσότερα \en{slices} ο πλακούντας καταλαμβάνει μικρό μέρος του όγκου, κάτι που
        δυσκολεύει τη μάθηση και φέρει την ανάγκη για στοχευμένη δειγματοληψία και μεθοδολογία (λεπτομερώς παρακάτω \ref{subsec:preproc}) \cite{73459184c1ce4a909dcc8bae5710cd1d,Kamnitsas_2017,Isensee_2020}.
  \item \textbf{Κίνηση} (μητέρα/εμβρύου) που οδηγεί σε \en{artifacts}, διαστρεβλώσεις και θολή εικόνα.
  \item \textbf{Ανομοιογένεια έντασης} (\en{bias field}) που επηρεάζει την εμφάνιση ίδιου ιστού σε διαφορετικές περιοχές.
\end{itemize}

\begin{figure}[!ht] \centering
\includegraphics[width=0.8\linewidth]{figures/First Slice along segmentation and 3D view.png}
\includegraphics[width=0.8\linewidth]{figures/Second Slice along segmentation and 3D view.png}
\includegraphics[width=0.8\linewidth]{figures/Third Slice along segmentation and 3D view.png}
 \caption{Αριστερά: ενδεικτική δισδιάστατη τομή (\en{slice}) \en{MRI} με την επισήμανση (πράσινο) του πλακούντα.
Δεξιά: τρισδιάστατη αναπαράσταση της αντίστοιχης μάσκας τμηματοποίησης (πράσινο) και της θέσης της τομής στον όγκο.}
\label{figure2.1}
\end{figure}

Σχετικές εργασίες έχουν δείξει ότι ακόμη και με προσεκτική χειροκίνητη τμηματοποίηση η διαδικασία είναι χρονοβόρα και
εμφανίζει ασυνέπεια ως προς διαφορετικούς παρατηρητές. Παράλληλα, έχουν παρουσιαστεί μέθοδοι αυτόματης/ημιαυτόματης τμηματοποίησης για
πλακούντα σε \en{MRI}, συμπεριλαμβανομένων πλαισίων που αντιμετωπίζουν δεδομένα με θόρυβο
\cite{alan._2016,Wang2016SlicSeg}
καθώς και προσεγγίσεων βαθιάς μάθησης για τρισδιάστατη τμηματοποίηση πλακούντα και εκτίμηση όγκου
\cite{shahedi2021placenta, shahedi2022automatic, liu2023evaluation}.

\section{Μαγνητική Τομογραφία \en{MRI}}

Η Μαγνητική Τομογραφία (\en{Magnetic Resonance Imaging, MRI}) είναι τεχνική απεικόνισης που βασίζεται στο φαινόμενο του
πυρηνικού μαγνητικού συντονισμού. Συνοπτικά, τα πρωτόνια (κυρίως του υδρογόνου) ευθυγραμμίζονται με το
στατικό μαγνητικό πεδίο, διεγεί\-ρονται με ραδιοσυχνότητες, και η επιστροφή τους στην ισορροπία παράγει σήμα που
καταγράφεται και αναδομείται σε εικόνα. Η αντίθεση μεταξύ ιστών εξαρτάται από παραμέτρους όπως οι χρόνοι χαλάρωσης
$T_1$ και $T_2$, καθώς και από τη διαμόρφωση των ακολουθιών λήψης. Αναλυτικότερη θεμελίωση παρέχεται σε κλασικά
συγγράμματα \cite{haacke1999mri,mcrobbie2006mri}.

Για την τμηματοποίηση, η μαγνητική παρουσιάζει πλεονεκτήματα (υψηλή αντίθεση μαλακών ιστών, \en{3D} πληροφορία), αλλά και
προκλήσεις: θόρυβο, ανομοιογένειες έντασης, και μεταβολές μεταξύ σαρωτών/πρωτοκόλλων. Ειδικά σε απεικόνιση κύησης, η
κίνηση αποτελεί συχνό πρόβλημα, επηρεάζοντας την ευκρίνεια και τη συνέπεια μεταξύ τομών
\cite{alan._2016}.

\subsection{Ανομοιογένεια έντασης και διόρθωση \en{bias field}}

Η ανομοιογένεια έντασης (\en{intensity non-uniformity}) ή \en{bias field} είναι αργά μεταβαλλόμενη παραμόρφωση που
οδηγεί σε διαφορετικές εντάσεις για τον ίδιο ιστό σε διαφορετικά σημεία του όγκου. Μία καθιερωμένη μέθοδος διόρθωσης
είναι ο αλγόριθμος \en{N4ITK} \cite{tustison2010n4itk}, ο οποίος βελτιώνει προγενέστερες προσεγγίσεις και χρησιμοποιείται
ευρέως ως προεπεξεργαστικό βήμα.

\section{Συνήθη βήματα προεπεξεργασίας για τμηματοποίηση \en{MRI}}
\label{subsec:preproc}

Αν και το ακριβές \en{pipeline} εξαρτάται από το πρόβλημα και το σύνολο δεδομένων, σε \en{3D MRI} τμηματοποίηση
εφαρμόζονται συχνά \cite{app14125144}:
\begin{itemize}
  \item \textbf{Εναρμόνιση προσανατολισμού} (κοινός άξονας αναφοράς).
  \item \textbf{Επαναδειγματοληψία} (\en{resampling}) σε επιθυμητή χωρική ανάλυση (\en{voxel spacing}).
  \item \textbf{Κανονικοποίηση έντασης} (κλιμάκωση σε $[0,1]$).
  \item \textbf{Περικοπή εικόνας} (π.χ. \en{crop foreground}) ώστε να μειωθεί το περιττό υπόβαθρο και η ανισορροπία κλάσεων.
  \item \textbf{Εμπλουτισμός δεδομένων} (\en{augmentation}) με τυχαίες περιστροφές/αντιστροφές/παραμορφώσεις/θόρυβο.
\end{itemize}

Στόχος των παραπάνω είναι (α) η διευκόλυνση της εκπαίδευσης με σταθερότερα στατιστικά,
(β) η κανονικοποίηση των δεδομένων και
(γ) η βελτίωση της γενίκευσης του μοντέλου.

\section{Μετρικές αξιολόγησης τμηματοποίησης}
\label{sec:metrics}

Για την αξιολόγηση δυαδικής τμηματοποίησης χρησιμοποιούνται μετρικές επικάλυψης μεταξύ πρόβλεψης $P$ και αλήθειας $G$.
Οι πιο διαδεδομένες είναι:

\subsection{Συντελεστής \en{Dice} (\en{DSC})}
Ο συντελεστής \en{Dice} ορίζεται ως:

\begin{equation} 
\mathrm{DSC}(P,G) = \frac{2|P \cap G|}{|P| + |G|}. 
\footnote{Ο τελεστής $\cap$ υπολογίζει την τομή δύο συνόλων $A$ και $B$ που αποτελέιται απο τα κοινά στοιχεία τους.}
\end{equation} 
Η τιμή του βρίσκεται στο $[0,1]$, όπου $1$ δηλώνει τέλεια επικάλυψη.

\subsection{\en{Intersection over Union} (\en{IoU}/\en{Jaccard})}
Ο δείκτης \en{Jaccard} ορίζεται ως:
\begin{equation}
\mathrm{IoU}(P,G) = \frac{|P \cap G|}{|P \cup G|}.
\end{equation}
Και εδώ οι τιμές είναι στο $[0,1]$. Οι δύο μετρικές σχετίζονται μονοτονικά και συχνά αναφέρονται μαζί.

\section{Σύνοψη}
\label{sec:ch2_summary}

Στο κεφάλαιο αυτό παρουσιάστηκαν:
\begin{itemize}
  \item βασικές έννοιες φυσιολογίας/ανάπτυξης του πλακούντα και η κλινική του σημασία,
  \item ο ρόλος της τμηματοποίησης ως υπολογιστικό πρόβλημα,
  \item οι ιδιαίτερες δυσκολίες της τμηματοποίησης πλακούντα σε \en{MRI},
  \item βασικές αρχές της \en{MRI} που σχετίζονται με τη φύση των δεδομένων,
  \item και οι μετρικές αξιολόγησης που χρησιμοποιούνται στην παρούσα εργασία.
\end{itemize}
  
