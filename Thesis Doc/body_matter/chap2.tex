\chapter{Θεωρητικό υπόβαθρο}
\InitialCharacter{Σ}το κεφάλαιο αυτό παρουσιάζονται  αναλυτικά οι τρεις
βασικές τεχνολογίες που έχουν σχέση με την εργασία αυτή, δηλαδή τα
συστήματα ομότιμων κόμβων, το πλαίσιο \en{RDF} και οι γλώσσες
ερωτήσεων για \en{RDF}.

\section{Αυτόματη τμηματοποίηση εικόνων}
\subsection{Τι είναι τμηματοποίηση}
 % Liga logia gia to segmentation xwris anafora se NNs etc
\section{Συνελικτικά Νευρωνικά Δίκτυα}
\subsection{Η μέθοδος της συνέλιξης}
 % explain convolution in deep learning
\subsection{Αρχιτεκτονική \en{UNet}}
 % explain UNet functionality quickly
\subsection{Περιορισμοί}
 % what are the general limitations with this type of architectures?
\section{\en{Transformers}}
\subsection{Συνοπτική εισαγωγή}
 % eisagogi stin theoria twn transformers
\subsection{Χρησιμότητα στην διαδικασία της τμηματοποίησης}
 % ti mas dinei poy den exoyme me ta CNNs
% pws kanoume cite \cite{elli05}.

Οι αρχές που διέπουν τα συστήματα ομότιμων κόμβων είναι οι εξής:
\begin{itemize}
\item Η αρχή του μοιράσματος των πόρων.
\item Η αρχή της αυτοοργάνωσης.
\end{itemize}

Σύμφωνα με το συντακτικό αυτό, το παράδειγμα γράφεται ως εξής:
\src{
\begin{tabbing}
1.<?x\=ml\= v\=ersion="1.0"?> \\
2.<rdf:RDFxmlns:rdf="http://www.w3.org/1999/02/22-rdf-syntax-ns\#" \\
3.\>\>\>xmlns:dc="http://purl.org/dc/elements/1.1/" \\
4.\>\>\>xmlns:exterms="http://www.example.org/terms/"> \\
5.\><rdf:Description
rdf:about="http://www.example.org/index.html"> \\
6.\>\><exterms:creation-date>August 16, 1999</exterms:creation-date> \\
7.\>\><dc:language>en</dc:language> \\
8.\>\><dc:creator rdf:resource="http://www.example.org/staffid/85740"/> \\
9.\></rdf:Description> \\
10.</rdf:RDF> \\
\end{tabbing}
}
\subsection{}
