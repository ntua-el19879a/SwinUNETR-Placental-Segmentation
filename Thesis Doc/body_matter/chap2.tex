%===============================
% chap2.tex (updated)
%===============================
\chapter{Τμηματοποίηση Ιατρικών Εικόνων και Μαγνητική Τομογραφία}
\label{ch:seg_mri}
\InitialCharacter{Ο} πλακούντας αποτελεί ένα \textit{παροδικό αλλά κρίσιμο} όργανο της κύησης, το οποίο λειτουργεί ως
διεπιφάνεια μεταξύ μητέρας και εμβρύου, επιτρέποντας την ανταλλαγή αερίων, θρεπτικών συστατικών και μεταβολιτών, ενώ
παράλληλα επιτελεί σημαντικούς ενδοκρινικούς και ανοσολογικούς ρόλους 
\cite{burton2015placenta,maltepe2015forgotten,gude2004growth}.
Η λειτουργία του πλακούντα σχετίζεται άμεσα με την ομαλή έκβαση της κύησης, και η πλακουντιακή δυσλειτουργία έχει
συσχετιστεί με σοβαρές μαιευτικές επιπλοκές, όπως προεκλαμψία και περιορισμό ενδομήτριας ανάπτυξης, στο πλαίσιο
διαταραχών της «βαθιάς πλακούντωσης» \cite{brosens2011deepplacentation}.

Σε αυτό το πλαίσιο, η ιατρική απεικόνιση -- και ειδικότερα η Μαγνητική Τομογραφία (\en{MRI}) -- μπορεί να υποστηρίξει
την \textit{ποσοτική} αξιολόγηση του πλακούντα. Ωστόσο, για να εξαχθούν αξιόπιστες μετρήσεις (π.\,χ. όγκος, σχήμα,
μορφολογικοί δείκτες ή/και χαρακτηριστικά υφής), απαιτείται πρώτα ο ακριβής διαχωρισμός του πλακούντα από το υπόβαθρο:
η \textbf{τμηματοποίηση} (\en{segmentation}). Στην παρούσα εργασία, στόχος είναι η τμηματοποίηση του πλακούντα σε
ογκομετρικά δεδομένα \en{MRI}, δηλαδή η παραγωγή μιας δυαδικής μάσκας (πλακούντας έναντι υποβάθρου) για κάθε εικόνα.

\section{Ο πλακούντας: ανατομία, ανάπτυξη και λειτουργίες}
\label{sec:placenta_phys}

Ο πλακούντας σχηματίζεται κατά την εμφύτευση και αναπτύσσεται με ταχύ ρυθμό κατά την κύηση, με σκοπό να υποστηρίξει τις
αυξανόμενες ανάγκες του εμβρύου. Από ανατομικής άποψης, διακρίνεται:
\begin{itemize}
  \item η \textbf{εμβρυϊκή πλευρά} (\en{chorionic plate}), από όπου εκφύονται οι χοριακές λάχνες και το δίκτυο αγγείων του
        εμβρύου,
  \item η \textbf{μητρική πλευρά} (\en{basal plate/decidua}), η οποία συνδέεται με το τοίχωμα της μήτρας,
  \item ο \textbf{μεσολάχνιος χώρος} (\en{intervillous space}), όπου κυκλοφορεί μητρικό αίμα και πραγματοποιείται η κύρια
        ανταλλαγή ουσιών.
\end{itemize}
Οι χοριακές λάχνες, επενδεδυμένες από τροφοβλαστικούς πληθυσμούς (με κυρίαρχο το \en{syncytiotrophoblast}), αποτελούν το
βασικό «λειτουργικό υπόστρωμα» ανταλλαγής \cite{gude2004growth,maltepe2015forgotten}.

Καθοριστικό βήμα στην ομαλή πλακουντοποίηση είναι η \textbf{αναδιαμόρφωση των σπειροειδών αρτηριών} της μήτρας, μέσω
εισβολής εξωλαχνικού τροφοβλάστη (\en{extravillous \- trophoblast}), ώστε να εξασφαλιστεί χαμηλής αντίστασης ροή προς τον
πλακούντα. Η ελαττωματική πλακουντοποίηση έχει προταθεί ως κοινός παθοφυσιολογικός μηχανισμός πίσω από τις λεγόμενες
\textit{μεγάλες μαιευτικές συνδρομές} (π.\,χ. προεκλαμψία, αποκόλληση πλακούντα, \en{fetal growth restriction})
\cite{brosens2011deepplacentation,maltepe2010placenta}.

Λειτουργικά, ο πλακούντας:
\begin{itemize}
  \item εξασφαλίζει \textbf{μεταφορά} οξυγόνου και θρεπτικών (γλυκόζη, αμινοξέα, λιπίδια) και \textbf{απομάκρυνση}
        μεταβολικών προϊόντων,
  \item δρα ως \textbf{ενδοκρινικό όργανο} (παραγωγή ορμονών που ρυθμίζουν την κύηση και τον μεταβολισμό της μητέρας),
  \item συμμετέχει στην \textbf{ανοσολογική ανοχή} και στη ρύθμιση της μητρο-εμβρυϊκής διεπιφάνειας.
\end{itemize}
Σύγχρονες ανασκοπήσεις τονίζουν ότι ο πλακούντας δεν είναι απλώς «φίλτρο», αλλά ένα πολυλειτουργικό όργανο με δυναμική
εξέλιξη και προσαρμογή κατά την κύηση \cite{burton2015placenta,maltepe2015forgotten}.

Η παραπάνω φυσιολογική πολυπλοκότητα εξηγεί γιατί η \textbf{ποσοτική} αξιολόγηση του πλακούντα (δομή/μορφολογία, χωρική
κατανομή, όγκος) αποτελεί ενεργό ερευνητικό πεδίο. Η \en{MRI} προσφέρει υψηλή αντίθεση μαλακών ιστών και τρισδιάστατη
πληροφορία, όμως η ποσοτικοποίηση προϋποθέτει αξιόπιστο \textit{ορισμό της περιοχής του πλακούντα} μέσα στον όγκο, δηλαδή
ακριβή τμηματοποίηση. Η ανάγκη αυτή αποτυπώνεται και σε μελέτες που αναπτύσσουν αυτοματοποιημένους αλγορίθμους για
τμηματοποίηση πλακούντα/μήτρας σε \en{MRI} και αναφέρουν άμεσα τη σχέση της τμηματοποίησης με μετρήσεις όγκου και
επαναληψιμότητα \cite{shahedi2021placenta}.

\section{Η τμηματοποίηση ως υπολογιστικό πρόβλημα}
\label{sec:seg_problem}

Σε ένα τυπικό σενάριο \textbf{δυαδικής τμηματοποίησης}, η είσοδος είναι ένας όγκος
$X \in \mathbb{R}^{H \times W \times D}$ (ή γενικότερα πολυκαναλικός όγκος) και η έξοδος είναι μια μάσκα
$Y \in \{0,1\}^{H \times W \times D}$ που υποδηλώνει για κάθε \en{voxel} αν ανήκει στην κλάση ενδιαφέροντος.
Η μάθηση διατυπώνεται συνήθως ως \textbf{επιβλεπόμενη} μάθηση: διαθέτουμε σύνολο ζευγών $(X_i, Y_i)$ και επιδιώκουμε
μοντέλο $f_\theta$ ώστε $f_\theta(X_i)\approx Y_i$.

Η τμηματοποίηση διαφέρει από την ταξινόμηση εικόνων σε δύο σημεία:
\begin{itemize}
  \item Η έξοδος είναι \textbf{πυκνή} πρόβλεψη (ανά \en{pixel/voxel}) και όχι μία ετικέτα ανά εικόνα.
  \item Η αβεβαιότητα εντοπίζεται συχνά στα \textbf{όρια} και επηρεάζεται από \en{partial volume effects}, θόρυβο και
        ανομοιογένειες έντασης, γεγονός που καθιστά την επιλογή μετρικών και συναρτήσεων κόστους κρίσιμη.
\end{itemize}

Στην \en{3D} τμηματοποίηση, οι υπολογιστικές απαιτήσεις αυξάνονται σημαντικά λόγω του πλήθους των \en{voxels}. Έτσι, συχνά
χρησιμοποιούνται στρατηγικές \textbf{\en{patch-based}} εκπαίδευσης και \textbf{\en{sliding-window inference}}
(\textit{στο Κεφάλαιο \ref{ch:ml_dl_models} αναλύονται οι αντίστοιχες πρακτικές}), έτσι ώστε να είναι
εφικτή η εκ\-παίδευση/πρόβλεψη σε μνήμη \en{GPU}.


\section{Ιδιαιτερότητες τμηματοποίησης πλακούντα σε \en{MRI}}
\label{sec:placenta_specifics}

Η τμηματοποίηση πλακούντα σε \en{MRI} είναι απαιτητική για λόγους που σχετίζονται τόσο με τη φυσιολογία όσο και με τη
διαδικασία λήψης:
\begin{itemize}
  \item \textbf{Μεγάλη μορφολογική μεταβλητότητα}: μέγεθος, σχήμα και θέση του πλακούντα μεταβάλλονται σημαντικά.
  \item \textbf{Περιορισμένη αντίθεση} σε ορισμένες ακολουθίες, ιδιαίτερα στα όρια πλακούντα-παρακείμενων ιστών.
  \item \textbf{Ανισορροπία κλάσεων}: στα περισσότερα \en{slices} ο πλακούντας καταλαμβάνει μικρό μέρος του όγκου, κάτι που
        δυσκολεύει τη μάθηση με τυπικές απώλειες.
  \item \textbf{Κίνηση} (μητέρα/εμβρύου) που οδηγεί σε \en{artifacts}, διαστρεβλώσεις και θολή εικόνα.
  \item \textbf{Ανομοιογένεια έντασης} (\en{bias field}) που επηρεάζει την εμφάνιση ίδιου ιστού σε διαφορετικές περιοχές.
\end{itemize}

\begin{figure}[!ht] \centering
\includegraphics[width=0.8\linewidth]{figures/First Slice along segmentation and 3D view.png}
\includegraphics[width=0.8\linewidth]{figures/Second Slice along segmentation and 3D view.png}
\includegraphics[width=0.8\linewidth]{figures/Third Slice along segmentation and 3D view.png}
 \caption{Αριστερά: ενδεικτική δισδιάστατη τομή (\en{slice}) \en{MRI} με την επισήμανση (πράσινο) του πλακούντα.
Δεξιά: τρισδιάστατη αναπαράσταση της αντίστοιχης μάσκας τμηματοποίησης (πράσινο) και της θέσης της τομής στον όγκο.}
\label{figure2.1}
\end{figure}

Σχετικές εργασίες έχουν δείξει ότι ακόμη και με προσεκτική χειροκίνητη τμηματοποίηση, η διαδικασία είναι χρονοβόρα και
εμφανίζει ασυνέπεια ως προς διαφορετικούς παρατηρητές. Παράλληλα, έχουν παρουσιαστεί μέθοδοι αυτόματης/ημιαυτόματης τμηματοποίησης για
πλακούντα σε \en{MRI}, συμπεριλαμβανομένων πλαισίων που αντιμετωπίζουν δεδομένα με κίνηση
\cite{alan._2016,Wang2016SlicSeg}
καθώς και προσεγγίσεων βαθιάς μάθησης για τρισδιάστατη τμηματοποίηση πλακούντα/μήτρας και εκτίμηση όγκου
\cite{shahedi2021placenta}.

\section{Μαγνητική Τομογραφία \en{MRI}}
\label{sec:mri_basics}

Η Μαγνητική Τομογραφία (\en{Magnetic Resonance Imaging, MRI}) είναι τεχνική απεικόνισης που βασίζεται στο φαινόμενο του
πυρηνικού μαγνητικού συντονισμού. Σε πολύ συνοπτικό επίπεδο, τα πρωτόνια (κυρίως του υδρογόνου) ευθυγραμμίζονται με το
στατικό μαγνητικό πεδίο, διεγείρονται με ραδιοσυχνότητες, και η επιστροφή τους στην ισορροπία παράγει σήμα που
καταγράφεται και αναδομείται σε εικόνα. Η αντίθεση μεταξύ ιστών εξαρτάται από παραμέτρους όπως οι χρόνοι χαλάρωσης
$T_1$ και $T_2$, καθώς και από τη διαμόρφωση των ακολουθιών λήψης. Αναλυτικότερη θεμελίωση παρέχεται σε κλασικά
συγγράμματα \cite{haacke1999mri,mcrobbie2006mri}.

Για την τμηματοποίηση, η \en{MRI} παρουσιάζει πλεονεκτήματα (υψηλή αντίθεση μαλακών ιστών, 3D πληροφορία), αλλά και
προκλήσεις: θόρυβο, ανομοιογένειες έντασης, και μεταβολές μεταξύ σαρωτών/πρωτοκόλλων. Ειδικά σε απεικόνιση κύησης, η
κίνηση αποτελεί συχνό πρόβλημα, επηρεάζοντας την ευκρίνεια και τη συνέπεια μεταξύ τομών
\cite{alan._2016}.

\subsection{Ανομοιογένεια έντασης και διόρθωση \en{bias field}}
\label{subsec:bias}

Η ανομοιογένεια έντασης (\en{intensity non-uniformity}) ή \en{bias field} είναι αργά μεταβαλλόμενη παραμόρφωση που
οδηγεί σε διαφορετικές εντάσεις για τον ίδιο ιστό σε διαφορετικά σημεία του όγκου. Μία καθιερωμένη μέθοδος διόρθωσης
είναι ο αλγόριθμος \en{N4ITK} \cite{tustison2010n4itk}, ο οποίος βελτιώνει προγενέστερες προσεγγίσεις και χρησιμοποιείται
ευρέως ως προεπεξεργαστικό βήμα.

\section{Συνήθη βήματα προεπεξεργασίας για τμηματοποίηση \en{MRI}}
\label{sec:preprocess}

Αν και το ακριβές \en{pipeline} εξαρτάται από το πρόβλημα και το σύνολο δεδομένων, σε 3D \en{MRI} τμηματοποίηση
εφαρμόζονται συχνά:
\begin{itemize}
  \item \textbf{Εναρμόνιση προσανατολισμού} (π.χ. κοινός άξονας αναφοράς).
  \item \textbf{Επανάληψη Δειγματοληψίας} (\en{resampling}) σε επιθυμητή χωρική ανάλυση (\en{voxel spacing}).
  \item \textbf{Κανονικοποίηση έντασης} (π.χ. κλιμάκωση σε $[0,1]$ με \en{percentiles}).
  \item \textbf{Περικοπή σε διαστάσεις \en{ROI}} (π.χ. \en{crop foreground}) ώστε να μειωθεί το περιττό υπόβαθρο.
  \item \textbf{Εμπλουτισμός δεδομένων} (\en{augmentation}) με τυχαίες περιστροφές/αντιστροφές/παραμορφώσεις/θόρυβο.
\end{itemize}

Στόχος των παραπάνω είναι (α) η διευκόλυνση της εκπαίδευσης με σταθερότερα στατιστικά,
(β) η κανονικοποίηση των δεδομένων και
(γ) η βελτίωση της γενίκευσης του μοντέλου.

\section{Μετρικές αξιολόγησης τμηματοποίησης}
\label{sec:metrics}

Για την αξιολόγηση δυαδικής τμηματοποίησης χρησιμοποιούνται μετρικές επικάλυψης μεταξύ πρόβλεψης $P$ και αλήθειας $G$.
Οι πιο διαδεδομένες είναι:

\subsection{Συντελεστής \en{Dice} (\en{DSC})}
Ο συντελεστής \en{Dice} ορίζεται ως:
\begin{equation}
\mathrm{DSC}(P,G) = \frac{2|P \cap G|}{|P| + |G|}.
\end{equation}
Η τιμή του βρίσκεται στο $[0,1]$, όπου $1$ δηλώνει τέλεια επικάλυψη.

\subsection{\en{Intersection over Union} (\en{IoU}/\en{Jaccard})}
Ο δείκτης \en{Jaccard} ορίζεται ως:
\begin{equation}
\mathrm{IoU}(P,G) = \frac{|P \cap G|}{|P \cup G|}.
\end{equation}
Και εδώ οι τιμές είναι στο $[0,1]$. Οι δύο μετρικές σχετίζονται μονοτονικά και συχνά αναφέρονται μαζί.

\section{Σύνοψη}
\label{sec:ch2_summary}

Στο κεφάλαιο αυτό παρουσιάστηκαν: (α) βασικές έννοιες φυσιολογίας/ανάπτυξης του πλακούντα και η κλινική του σημασία,
(β) ο ρόλος της τμηματοποίησης ως υπολογιστικό πρόβλημα, (γ) οι ιδιαίτερες δυσκολίες της τμηματοποίησης πλακούντα σε
\en{MRI}, (δ) βασικές αρχές της \en{MRI} που σχετίζονται με τη φύση των δεδομένων, και (ε) οι μετρικές αξιολόγησης που
χρησιμοποιούνται στην παρούσα εργασία.
